\textbf{Problema 2.} (5 puntos)

\vspace{20px}
\textit{Solución:}
\\

\begin{enumerate}
[label=\alph*)]
    \item La función $f(z) = \sqrt[5]{z}$, en coordenadas polares, con corte a lo largo del eje $x$ negativo, tiene la siguiente expresión:

    \begin{equation*}
        f(z) = \sqrt[5]{r} \exp{\Bigl(i\frac{\theta}{5}\Bigl)} \hspace{20pt} (r > 0, -\pi < \theta < \pi),
    \end{equation*}

    a la que se ha podido llegar mediante la igualdad $\sqrt[5]{z} = \exp(\frac{1}{5} \log{z})$.

    Cabe remarcar que los puntos situados a lo largo del corte son
    puntos singulares, en los que $f(z)$ no está definida.

    Podemos demostrar que esta función $f(z) = u(r,\theta) + i v(r,\theta)$ es analítica
    en su dominio demostrando que las derivadas parciales de $u$ y $v$ existen en su dominio y que satisfacen la forma polar
    de las ecuaciones de Cauchy-Riemann:

    \begin{equation*}
        r u_r = v_\theta, \hspace{20pt} u_\theta = - r v_r
    \end{equation*}

    Como $u(r,\theta) = \sqrt[5]{r} \cos{\frac{\theta}{5}}$ y $v(r,\theta) = \sqrt[5]{r} \sen{\frac{\theta}{5}}$:

    \begin{align*}
        u_r  = & \;  \frac{1}{5} r^{-4/5} \cos{\frac{\theta}{5}} \\
        v_\theta  = & \;  \frac{1}{5} r^{1/5} \cos{\frac{\theta}{5}} \\
        u_\theta  = & \;  - \frac{1}{5} r^{1/5} \sen{\frac{\theta}{5}} \\
        v_r  = & \;  \frac{1}{5} r^{-4/5} \sen{\frac{\theta}{5}} \\
    \end{align*}

    Por ello, las ecuaciones de Cauchy-Riemann se cumplen y hemos demostrado que $f(z)$ es analítica en su dominio.


    \vspace{20px}
    \item Comenzamos calculando una primitiva.

    Sabemos que si una función $f(z)$ es continua en un dominio $D$, y si a su vez se puede calcular una primitiva $F(z)$ en $D$,
    esto implica que las integrales de $f(z)$, a lo largo de contornos contenidos enteramente en $D$, y extendiendo de cualquier punto
    fijo $z_1$ a otro punto fijo $z_2$, tienen todas el mismo valor, independientemente del contorno recorrido:

    \begin{equation*}
        \int_{z_1}^{z_2} f(z)dz = F(z_2) - F(z_1)
    \end{equation*}

    Nos damos cuenta también de que una primitiva $F(z)$ de la determinación principal de la raíz $\sqrt[5]{z}$ no es diferenciable,
    ni siquiera está definida a lo largo del corte, por lo que el contorno $C$ no está contenido en ningún dominio donde
    esa posible primitiva cumpla $F^\prime(z) = \sqrt[5]{z}$.

    Por todo ello, debemos usar una combinación de dos primitivas para calcular la integral de $f(z)$ a lo largo de $C$.

    Consideramos primero la rama:

    \begin{equation*}
        f_1(z) = \sqrt[5]{r} e^{i\frac{\theta}{5}}, \hspace{12pt} (r> 0, \frac{\pi}{2} < \theta < \frac{5\pi}{2})
    \end{equation*}

    Evaluamos la integral de esta rama a lo largo de cualquier contorno $C_1$ que vaya desde $z = -1$ hasta $z = 1$ que,
    salvo sus extremos, esté situado por debajo del eje real.

    \begin{equation*}
        \int_{C_1} f_1(z) dz = \frac{5}{6} r^{\frac{6}{5}} e^{i\frac{6\theta}{5}} \Big|_\pi^{2\pi} =
        \frac{5}{6}\Bigl(e^{i\frac{12\pi}{5}} - e^{i\frac{6\pi}{5}} \Bigr)
    \end{equation*}

    A continuación, consideramos la rama:

    \begin{equation*}
        f_2(z) = \sqrt[5]{r} e^{i\frac{\theta}{5}}, \hspace{12pt} (r> 0, - \frac{\pi}{2} < \theta < \frac{3\pi}{2})
    \end{equation*}

    Esta vez, evaluamos la integral de esta rama a lo largo de cualquier contorno $C_2$ que vaya desde $z = -1$ hasta $z = 1$ y que,
    salvo sus extremos, esté situado por encima del eje real.

    \begin{equation*}
        \int_{C_2} f_2(z) dz = \frac{5}{6} r^{\frac{6}{5}} e^{i\frac{6\theta}{5}} \Big|_\pi^{0} =
        \frac{5}{6}\Bigl(e^{i0} - e^{i\frac{6\pi}{5}} \Bigr)
    \end{equation*}

    Como queremos recorrer $C$ en sentido antihorario, la integral final que buscamos se calcula evaluando $C_1 - C_2$:

    \begin{equation*}
        \int_{C} f(z) dz = \frac{5}{6} \Bigl( e^{i\frac{2\pi}{5}} -1 \Bigr) =
        \frac{5}{6} \Bigl(  \cos{\frac{2\pi}{5}} - 1 + i \sen{\frac{2\pi}{5}} \Bigr)
    \end{equation*}


    \vspace{20px}
    En este punto, en el enunciado se nos pide aplicar el teorema de Cauchy-Goursat para calcular la integral de $f(z)$
    y comprobar que sale el mismo resultado.

    Sin embargo, el teorema de Cauchy-Goursat no se puede aplicar con estas condiciones
    porque $f(z)$ no es analítica en el origen, y ni siquiera está
    definida en el eje real negativo.

    Tampoco se puede trazar un contorno $C$ que bordee el corte de la rama y hacer una elección de $f(z)$ uniforme o unívoca.

    Por último, no se puede aplicar teoría de residuos, porque para ello los puntos singulares interiores a $C$ deberían
    ser finitos en número y singulares aislados. Los puntos del corte de rama no cumplen ninguna de las dos condiciones.


\end{enumerate}