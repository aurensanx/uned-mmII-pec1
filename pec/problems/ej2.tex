\textbf{E2.} (3 puntos)


\vspace{20px}
\textit{Solución:}
\\

\begin{enumerate}
[label=\alph*)]
    \item La polarización \textbf{P} se define como el momento dipolar por unidad de volumen cuando dicho volumen es muy pequeño:

    \begin{equation*}
        \mathbf{P} = \lim_{\Delta v \rightarrow 0} \frac{\Delta \mathbf{p}}{\Delta v}
    \end{equation*}

    $\Delta \mathbf{p}$ es la suma vectorial de todos los momentos dipolares que existen en el volumen elemental $\Delta v$.
    La relación anterior en forma diferencial para un elemento de volumen $dv^\prime$ se escribe como:

    \begin{equation*}
        d\mathbf{p} = \mathbf{P}(\mathbf{r}^\prime) dv^\prime
    \end{equation*}

    Si integramos la expresión anterior en el volumen de la esfera obtendremos el momento dipolar buscado.

    \begin{equation*}
        \mathbf{p} = \int_{V} \mathbf{P}(\mathbf{r}^\prime) dv^\prime
        = \int_{V} P \mathbf{u}_z dv^\prime = \frac{4}{3} \pi a^3 P \mathbf{u}_z
    \end{equation*}

    \vspace{20px}
    \item Como se muestra en el ejemplo 5.2, el campo \textbf{E} en cualquier punto interior de una esfera con polarización
    $P \mathbf{u}_z $ es:

    \begin{equation*}
        \mathbf{E} = - \frac{P}{3 \varepsilon_0} \mathbf{u}_z
    \end{equation*}

    Esto tiene sentido físico, ya que si la polarización tiene sentido positivo en el eje $\mathbf{u}_z$, quiere decir
    que cargas positivas se van acumulando en la parte superior de la esfera, y cargas negativas en la inferior.
    El campo eléctrico siempre sigue la dirección de la acumulación de carga positivas a negativas.

    El vector desplazamiento eléctrico \textbf{D} se puede obtener mediante la ecuación $\mathbf{D} = \varepsilon_0 \mathbf{E} + \mathbf{P}$.


    \begin{equation*}
        \mathbf{D} = \varepsilon_0 \biggl( - \frac{\mathbf{P}}{3\varepsilon_0} \biggr) + \mathbf{P} = \frac{2}{3} P \mathbf{u}_z
    \end{equation*}

    \vspace{20px}
    \item
    La ecuación 4.7 es una forma de expresar el campo debido a un dipolo eléctrico.

    \begin{equation*}
        \mathbf{E} = \frac{1}{4\pi\varepsilon_0} \frac{1} {r^3} \biggl(
        \frac{3 (\mathbf{r} \cdot \mathbf{p})}{r^2} \mathbf{r} - \mathbf{p}
        \biggr)
    \end{equation*}

    En coordenadas esféricas, $\mathbf{r} = r \mathbf{u}_r$, y aplicando la transformación
    $\mathbf{u}_z = \mathbf{u}_r \cos \theta - \mathbf{u}_\theta \sen\theta$, el
    momento dipolar calculado en el apartado a) se expresa como:

    \begin{equation*}
        \mathbf{p} = \frac{4}{3} \pi a^3 P (\cos\theta \mathbf{u}_r - \sen\theta \mathbf{u}_\theta )
    \end{equation*}

    Sutituyendo estos valores en la ecuación del campo eléctrico, llegamos a la ecuación:


    \begin{equation*}
        \mathbf{p} = \frac{4}{3} \pi a^3 P (\cos\theta \mathbf{u}_r - \sen\theta \mathbf{u}_\theta )
    \end{equation*}

    Insertando estas expresiones en la ecuación del campo eléctrico:

    \begin{equation*}
        \mathbf{r} \cdot \mathbf{p} = (r \mathbf{u}_r) \cdot
        (\frac{4}{3} \pi a^3 P \mathbf{u}_z)
        = \frac{4}{3} \pi a^3 r P ( \mathbf{u}_r \cdot \mathbf{u}_z ) =
        \frac{4}{3} \pi a^3 r P \cos\theta
    \end{equation*}

    \begin{equation*}
        \frac{3(   \mathbf{r} \cdot \mathbf{p} )} {r^2} \mathbf{r} =
        4 \pi a^3 P \cos\theta  \mathbf{u}_r
    \end{equation*}

    \begin{equation*}
        \biggl( \frac{3(   \mathbf{r} \cdot \mathbf{p} )} {r^2} \mathbf{r} - \mathbf{p} \biggr) =
        \frac{4}{3} \pi a^3 P (
        2\cos\theta  \mathbf{u}_r - \sen\theta  \mathbf{u}_\theta
        )
    \end{equation*}

    \begin{equation*}
        \mathbf{E} = \frac{a^3 P}{3 \varepsilon_0 r^3} (
        2\cos\theta  \mathbf{u}_r - \sen\theta  \mathbf{u}_\theta
        )
    \end{equation*}

    La última expresión es el campo eléctrico creado por el dipolo puntual para $r > a$. Solo tiene componentes radial y azimutal en
    coordenadas esféricas ya que el dipolo está orientado a lo largo del eje Z en coordenadas cartesianas.

    La condición de frontera para el campo eléctrico en la superficie de separación entre dos medios es la continuidad
    de sus componentes tangenciales.

    En $r = a$, la frontera de la esfera, tenemos que el campo calculado a partir del la ecuación del dipolo puntual es:
    \begin{equation*}
        \mathbf{E} = \frac{P}{3 \varepsilon_0} (
        2\cos\theta  \mathbf{u}_r - \sen\theta  \mathbf{u}_\theta
        )
    \end{equation*}

    La componente $\mathbf{u}_r$ es perpendicular a la superficie de la esfera y la componente $\mathbf{u}_\theta$ es tangencial, así
    que para la condición de frontera solo nos interesa la contribución de la segunda.

    Nos damos cuenta de que la componente tangencial del campo eléctrico que habíamos calculado en el apartado b)
    para el interior de la esfera es la misma
    que la componente tangencial para el campo externo que hemos calculado a partir del valor del momento dipolar:

    \begin{equation*}
        \mathbf{E} = - \frac{P \sen\theta}{3 \varepsilon_0} \mathbf{u}_\theta
    \end{equation*}

    Por todo ello, hemos demostrado que los campos eléctricos interior y exterior a la esfera cumplen la condición de frontera.


    \vspace{20px}
    \item

    El material dieléctrico de la esfera es isótropo, homogéneo y lineal, así que la susceptibilidad $\chi$ no depende del campo
    eléctrico ni del punto considerado.

    Podemos escribir entonces la polarización a partir de su relación constitutiva con el campo eléctrico:

    \begin{equation*}
        \mathbf{P} = \varepsilon_0 \chi \mathbf{E} = \varepsilon_0 \chi E_0 \mathbf{u}_z
    \end{equation*}

    Utilizando el resultado del apartado b), sabemos que esta polarización crea un campo en el interior de la esfera igual a:

    \begin{equation*}
        \mathbf{E^\prime} = - \frac{P}{3 \varepsilon_0} = - \frac{\chi E_0}{3} \mathbf{u}_z
    \end{equation*}

    Sumando este campo eléctrico al que existía previamente a la introducción de la esfera, el campo \textbf{E} final dentro
    de la esfera es:

    \begin{equation*}
        \mathbf{E} = E_0 \Bigl( 1 - \frac{\chi}{3} \Bigr) \mathbf{u}_z
    \end{equation*}

    El campo \textbf{D} se calcula como:

    \begin{equation*}
        \mathbf{D} = \varepsilon_0 \mathbf{E} + \mathbf{P} =
        \varepsilon_0 E_0 \Bigl( 1 - \frac{\chi}{3} \Bigr) \mathbf{u}_z +  \varepsilon_0 E_0 \chi \mathbf{u}_z =
        \varepsilon_0 E_0 \Bigl( 1 + \frac{2}{3} \chi \Bigr) \mathbf{u}_z
    \end{equation*}


    En el límite $\chi \simeq 0$ no existe polarización, independientemente del campo eléctrico que exista previamente.

    En el límite $\chi \gg 0$, la polarización de la esfera compensa por completo el campo eléctrico previo en el interior de la esfera. Utlizando las
    relaciones anteriores, el valor de la polarización para este caso sería:

    \begin{equation*}
        \mathbf{P} = 3 \varepsilon_0 E_0 \mathbf{u}_z
    \end{equation*}

\end{enumerate}