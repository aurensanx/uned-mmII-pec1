\textbf{E1.} (2,5 puntos)

\usetikzlibrary{calc,patterns,angles,quotes}

\tikzset{
    cross/.pic = {
        \draw(-#1,0) -- (#1,0);
        \draw(0,-#1) -- (0, #1);
    }
}

\vspace{20px}
\textit{Solución:}
\\

\begin{enumerate}
[label=\alph*)]
    \item Para resolver este problema se aplica el principio de superposición lineal. El problema se reduce así
    a sumar las contribuciones de una esfera con densidad $\rho_0$ y centro en O, y de otra esfera
    con densidad $-\rho_0$ y centro en O$^\prime$.

    Como los puntos A y B se sitúan en la superficie de las 2 esferas, se podría calcular el campo \textbf{E} mediante
    la aplicación de la ley de Gauss, o suponiendo que toda la carga se sitúa en el centro de la esfera y
    utilizando la expresión del campo eléctrico para una carga discreta.

    Con ambos métodos, los resultados para la esfera 1 (centrada en O) y para la esfera 2 (centrada en O$^\prime$) son:

    \begin{equation*}
        \mathbf{E}_1 = \frac{\rho_0 R}{3 \varepsilon_0} \mathbf{u}_{\rho}; \hspace{20pt}
        \mathbf{E}_2 = - \frac{\rho_0 R}{3 \varepsilon_0} \mathbf{u}_{\rho}
    \end{equation*}

    Solo existe componente radial del campo eléctrico para cada una de las esferas. Observando la simetría del
    problema, podemos expresar los campos eléctricos en los puntos A y B en coordenadas cartesianas:

    \begin{equation*}
        \mathbf{E}_A =  \mathbf{E}_B = \frac{\rho_0 R}{3 \varepsilon_0} (\mathbf{u}_y + \mathbf{u}_z )
    \end{equation*}


    \vspace{20px}
    \item La dirección del campo en los puntos del segmento AB es la dirección del vector unitario $\frac{1}{\sqrt{2}}(\mathbf{u}_y + \mathbf{u}_z)$.
    Esta es la misma dirección que la obtenida para el campo en los
    puntos A y B en el apartado a).

    Mostramos en la siguiente figura el campo para un punto cualquiera del segmento:

    \begin{center}
        \begin{tikzpicture}
            \coordinate (orig1) at (0,0);
            \coordinate (orig2) at (4,4);
            \coordinate (point) at (1,3);
            \coordinate (a1) at (1,0);
            \coordinate (a2) at (0.25,0.75);
            \coordinate (b1) at (3,3.75);
            \coordinate (b2) at (1,4);

            \draw[->] (orig1) -- (5,0) node[right]{$Y$};
            \draw[->] (orig1) -- (0,5) node[above]{$Z$};

            \draw[dashed] (0,4)--(4,0);

            \draw (orig1) circle (0pt) node[above left]{O};
            \draw (orig2) pic {cross=4pt};
            \draw (orig2) circle (0pt) node[above right]{O$^\prime$};


            \draw[line width=2pt,blue,-stealth](orig1)--(1,3) node[below left=4pt]{$\mathbf{E}_A$};
            \draw[line width=2pt,red,-stealth](1,3)--(orig2) node[left=16pt]{$\mathbf{E}_B$};
            \draw[line width=1pt,blue!45,-stealth](0, 0)--(0,3) node[anchor=south];
            \draw[line width=1pt,blue!45,-stealth](0, 0)--(1,0) node[anchor=south];
            \draw[line width=1pt,red!45,-stealth](1, 3)--(1,4) node[anchor=south];
            \draw[line width=1pt,red!45,-stealth](1, 3)--(4,3) node[anchor=south];

%            \draw[dashed] (0,3)--(1,3);
%            \draw[dashed] (1,0)--(1,3);
%            \draw[dashed] (4,3)--(orig2);
%            \draw[dashed] (1,4)--(orig2);


            \pic [draw, -, "$\alpha$", angle eccentricity=1.5] {angle = a1--orig1--a2};
            \pic [draw, -, "$\alpha$", angle eccentricity=1.5] {angle = b1--point--b2};

        \end{tikzpicture}
    \end{center}

    Podemos observar por la descomposición en las componentes cartesianas de cada vector de campo eléctrico que el módulo
    del campo total en la dirección $\mathbf{u}_y$ siempre es igual al campo en la dirección $\mathbf{u}_z$.



    \vspace{20px}
    \item Sea $d$ la distancia desde el centro de cualquiera de las dos esferas a un punto $P$ situado en el segmento AB.

    Utilizando el principio de superposición lineal, podemos calcular el campo total en el punto $P$ con la suma de las contribuciones
    al campo de cada esfera. Fijándonos en la figura del apartado anterior, podemos escribir:

    \begin{equation*}
        \mathbf{E} = \mathbf{E}_A + \mathbf{E}_B = \frac{\rho_0 d}{3 \varepsilon_0} (\cos\alpha \mathbf{u}_y + \sen\alpha \mathbf{u}_z)
        + \frac{\rho_0 d}{3 \varepsilon_0} (\sen\alpha \mathbf{u}_y + \cos\alpha \mathbf{u}_z) =
        \frac{\rho_0}{3 \varepsilon_0} d (\cos\alpha + \sen\alpha) (\mathbf{u}_y + \mathbf{u}_z)
    \end{equation*}

    Por la simetría del problema, sabemos que $d (\cos\alpha + \sen\alpha) = R$, así que la expresión del campo en cualquier punto
    $P$ del segmento AB es:

    \begin{equation*}
        \mathbf{E} =  \frac{\rho_0 R}{3 \varepsilon_0} (\mathbf{u}_y + \mathbf{u}_z)
    \end{equation*}

    Este es un valor constante, que tiene como módulo $E = \frac{\sqrt{2}}{3} \frac{\rho_0 R}{\varepsilon_0}$, y la dirección del vector
    unitario indicado en el apartado b).

\end{enumerate}