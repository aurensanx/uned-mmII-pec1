\textbf{Problema 1.} (5 puntos)

\vspace{20px}
\textit{Solución:}
\\

\begin{enumerate}
[label=\alph*)]
    \item Una función real $H$, de variables reales $x$ e $y$, es armónica en un dominio dado del plano $xy$ si,
    en todo el dominio,
    tiene derivadas parciales continuas de primer y segundo orden que satisfacen la siguiente ecuación diferencial, conocida
    como ecuación de Laplace:

    \begin{equation*}
        H_{xx}(x,y) + H_{yy}(x.y) = 0
    \end{equation*}

    La función $u$ del enunciado es derivable en A (plano $xy$ excepto el origen) porque la única singularidad de $U$ o sus
    derivadas parciales es el origen.

    Solo nos queda calcular las derivadas parciales $u_{xx}$ e $u_{yy}$ y comprobar que su suma es 0.

    \begin{align*}
        u_{x}(x,y) = & \;1 + \frac{x^2 + y^2 - 2x^2}{(x^2 + y^2)^2} = 1 + \frac{y^2 - x^2}{(x^2 + y^2)^2} \\[20pt]
        u_{xx}(x,y) = & \;\frac{-2x(x^2 + y^2)^2 - (y^2 - x^2)2(x^2 + y^2)2x}{(x^2 + y^2)^4} =
        \frac{-2x(x^2 + y^2) - 4x(y^2 - x^2)}{(x^2 + y^2)^3}  \\
        = & \; \frac{-2x^3 -2xy -4xy^2 + 4x^3}{(x^2 + y^2)^3} =
        \frac{2x^3 - 6xy^2}{(x^2 + y^2)^3} \\[20pt]
        u_{y}(x,y) = & \; - \frac{2xy}{(x^2 + y^2)^2} \\[20pt]
        u_{yy}(x,y) = & \; - \frac{2x(x^2 + y^2)^2 - 2xy2(x^2+y^2)2y}{(x^2 + y^2)^4}
        = - \frac{2x(x^2 + y^2) - 8xy^2}{(x^2 + y^2)^3} = - \frac{2x^3 - 6xy^2}{(x^2+y^2)^3}
    \end{align*}

    Por lo tanto, $u_{xx}(x,y) + u_{yy}(x,y)$ es evidentemente $0$ y hemos verificado que $u(x,y)$ es armónica.


    \vspace{20px}
    \item Se nos dice en el enunciado que $Re[f(z)] = u(x,y)$. Podemos escribir por tanto la función compleja $f(z)$ como $f(z) = u(x,y) + iv(x,y)$,
    donde tanto $u$ como $v$ son funciones reales.

    Por otro lado, una función $f(z) = u(x,y) + iv(x,y)$ es analítica en un dominio $D$ si y solo si $v$ es conjugada
    armónica de $u$. $v$ es conjugada armónica de $u$ en un dominio $D$
    si ambas son funciones armónicas en $D$ y sus derivadas parciales de primer orden
    satisfacen las ecuaciones de Cauchy-Riemann en $D$:

    \begin{equation*}
        u_x = v_y,\hspace{12pt} u_y = -v_x
    \end{equation*}

    Hemos verificado en el apartado a) que $u(x,y)$ era una función armónica en el dominio $A$, por lo que podemos utilizar las ecuaciones de
    Cauchy-Riemann para calcular $v(x,y)$.

    Comenzamos escribiendo:

    \begin{equation*}
       v_x = -u_y = \frac{2xy}{(x^2+ y^2)^2}
    \end{equation*}

    Integrando a cada lado con respecto a $x$:

    \begin{equation*}
        v(x,y) = - \frac{y}{x^2 + y^2} + \phi(y),
    \end{equation*}

    donde $\phi(y)$ es, por el momento, una función arbitraria de $y$. Utilizando la otra ecuación de Cauchy-Riemann tenemos que:

    \begin{equation*}
        1 + \frac{y^2 - x^2}{(x^2+y^2)^2} = - \frac{(x^2+y^2)-y2y}{(x^2+y^2)^2} + \phi^{\prime}(y) = \frac{y^2 - x^2}{(x^2+y^2)^2} + \phi^{\prime}(y),
    \end{equation*}

    por lo que $\phi(y) = y + C$, siendo $C$ una constante arbitraria real. Se puede escribir entonces $v(x,y)$ como:

    \begin{equation*}
       v(x,y) = y - \frac{y}{x^2+y^2} + C
    \end{equation*}

    Se puede ver fácilmente por simetría respecto a $u$ que $v$ también es armónica.

    Hasta aquí, la función analítica correspondiente es:

    \begin{equation*}
        f(z) = x + \frac{x}{x^2+y^2} + i \Bigl(y - \frac{y}{x^2+y^2} + C\Bigr)
    \end{equation*}

    En este punto, aplicamos la segunda condición del enunciado, que nos dice que cuando $z=x$, $f(z)\,\epsilon\,\mathbb{R}$ o, equivalentemente,
    $v(x,0) = 0$, por lo que $C = 0$ y:


    \begin{equation*}
        f(z) = x + \frac{x}{x^2+y^2} + i \Bigl(y - \frac{y}{x^2+y^2}\Bigr)
    \end{equation*}

    Debemos dar esta solución en términos de z, por lo que reordenando términos y operando:


    \begin{equation*}
        f(z) = x + i y + \frac{x - iy}{x^2+y^2} = z + \frac{\bar{z}}{z\bar{z}} = z + \frac{1}{z}.
    \end{equation*}


    \vspace{20px}
    \item La curva $z = e^{it}$, con $\bigl(- \frac{\pi}{2} \leq t \leq 0 \bigr)$, en el plano $z$, es
    el círculo unidad en el cuarto cuadrante, recorrido en sentido antihorario:

    \begin{center}
        \begin{tikzpicture}

            \draw[-] (0,-2) -- (0,1) node[right]{$y$};
            \draw[-] (-1,0) -- (2,0) node[above]{$x$};
            \draw (0,0) circle (0pt) node[above left]{$O$};

%            \draw ($(1,0)+(0,2.5pt)$) -- (1,0);
            \node at ($(1,0)+(0,2ex)$) {1};

%            \draw ($(0,-1)-(2.5pt,0)$) -- (0,-1);
            \node at ($(0,-1)+(-2ex,0)$) {1};

            \draw[ line width=0.25mm,
                decoration={markings, mark=at position 0.625 with {\arrow[scale=1.5,>=stealth]{>}}},
                postaction={decorate}
            ]
            (0,-1) arc[start angle=-90, end angle=0, radius=1];
        \end{tikzpicture}
    \end{center}

    Debemos hallar y dibujar la imagen en el plano $w$ o plano $uv$ bajo la transformación $w= f(z) = z + \frac{1}{z}$.

    Sustituyendo, llegamos a $w = 2 \cos{t} $, con $\bigl(- \frac{\pi}{2} \leq t \leq 0 \bigr)$, por lo que:

    \begin{equation*}
       u = 2 \cos{t},\hspace{12pt} v = 0
    \end{equation*}

    Vemos que la imagen de la transformación es una recta que coincide con el eje de abscisas, recorrida desde 0 hasta 2:


    \begin{center}
        \begin{tikzpicture}

            \draw[-] (0,-1) -- (0,1) node[right]{$v$};
            \draw[-] (-1,0) -- (3,0) node[above]{$u$};
            \draw (0,0) circle (0pt) node[above left]{$O$};

            \draw ($(2,0)+(0,2.5pt)$) -- (2,0);
            \node at ($(2,0)+(0,2ex)$) {2};

            \draw[-, line width=0.25mm,
                decoration={markings, mark=at position 0.625 with {\arrow[scale=1.5,>=stealth]{>}}},
                postaction={decorate}
            ]
            (0,0) -- (2,0)
        \end{tikzpicture}
    \end{center}


\end{enumerate}